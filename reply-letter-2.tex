\documentclass{letter}
\usepackage[utf8]{inputenc}
\usepackage[T1]{fontenc}
\usepackage[a4paper,margin=3cm,top=2.5cm]{geometry}
\usepackage{microtype}
\usepackage{csquotes}
\linespread{1.3}
% Packages
\usepackage[dvipsnames,svgnames]{xcolor} % text color
\usepackage[normalem]{ulem} % wavy underlines

% Comments
\newcommand{\todo}[1]{\noindent\textcolor{red}{{\bf \{TODO}: #1{\bf \}}}}
\newcommand{\TODO}[1]{\todo{#1}}
\newcommand{\citeneeded}{\textcolor{red}{{\bf [?!]}}}
\newenvironment{draft}{\color{gray}}{\color{black}}

% Reviewers
\newcommand\rv[1]{{\color{RubineRed}\textbf{RV}: #1}}

% Annotations
\makeatletter
\font\uwavefont=lasyb10 scaled 700
\def\spelling{\bgroup\markoverwith{\lower3.5\p@\hbox{\uwavefont\textcolor{Red}{\char58}}}\ULon}
\def\grammar{\bgroup\markoverwith{\lower3.5\p@\hbox{\uwavefont\textcolor{LimeGreen}{\char58}}}\ULon}
\def\phrasing{\bgroup\markoverwith{\lower3.5\p@\hbox{\uwavefont\textcolor{RoyalBlue}{\char58}}}\ULon}
\let\rephrase\phrasing
\newcommand\remove{\bgroup\markoverwith{\textcolor{red}{\rule[0.5ex]{2pt}{0.4pt}}}\ULon}
\makeatother


\newcounter{section}
\newcounter{subsection}[section]
\setcounter{secnumdepth}{3}
\makeatletter
\newcommand\section{\@startsection {section}{1}{\z@}%
                                   {-3.5ex \@plus -1ex \@minus -.2ex}%
                                   {2.3ex \@plus.2ex}%
                                   {\normalfont\Large\bfseries}}
\newcommand\subsection{\@startsection{subsection}{2}{\z@}%
                                     {-3.25ex\@plus -1ex \@minus -.2ex}%
                                     {1.5ex \@plus .2ex}%
                                     {\normalfont\large\bfseries}}
\renewcommand \thesection{\@arabic\c@section}
\renewcommand\thesubsection{\thesection.\@arabic\c@subsection}
\makeatother

\signature{Ruben Taelman\\on behalf of the authors}
\address{imec -- Ghent University -- IDLab\\Technologiepark-Zwijnaarde 15\\B-9052 Gent\\Belgium}
\begin{document}

\begin{letter}{To the Editors of the Special Issue on\\Managing the Evolution and Preservation of the Data Web\\of the Journal of Web Semantics}

\opening{Dear Editors and Reviewers,}

\bigskip

Thank you for your reviews.
Please find attached to this letter a revised version of our submission entitled
\enquote{Triple Storage for Random-Access Versioned Querying of RDF Archives}
in which we have addressed your comments.

\bigskip

Hereafter, you can find detailed answers to your questions and remarks.
Should you have any further questions or concerns, please do not hesitate to contact us.
We look forward to hearing back from you and thank you in advance.

\bigskip

\closing{Sincerely,}

\pagebreak
\section*{Review 2}

\textbf{\enquote{Missing overview of related work in the context of evolving graph databases (not just RDF).}}

Non-RDF graph databases do not offer native versioning capabilities to the best of our knowledge.
We now briefly mention this lack in the related work section.

\textbf{\enquote{Two recent papers on Extended Characteristic Sets are missing}}

We now briefly discuss this topic and the mentioned papers.

\section*{Review 3}

\textbf{\enquote{The BEAR journal paper has been accepted at Semantic Web Journal}}

We have updated the BEAR reference.

\textbf{\enquote{Strange notation in formalization}}

We improved the notation of the symbols in the formalization in Section 2.4.

\textbf{\enquote{In the problem statement in Section 3, I would maybe briefly mention the meaning of "offset" in this context}}

Offset and limit have now been clarified in Section 3.

\textbf{\enquote{The titles of Sections 4, 5, 6 and 7 are too generic}}

In our opinion, the title of Section 4 (Overview: Storage and Querying) is concrete enough for its contents.
We have updated the titles of Sections 5, 6 and 7.

\textbf{\enquote{I would suggest to represent the triples with the notation (D0, S1, D1)}}

We have updated all triples in the paper to use the suggested notation.

\textbf{\enquote{Figure 3 and 4 could be somehow aligned}}

We have updated Figure 4 to be aligned with Figure 3.
The representations of the addition and deletion trees have been made consistent now.

\textbf{\enquote{Results are not commented in the result section}}

All results are now commented on in the results section.
They are then summarized and discussed later.

\section*{Review 4}

\textbf{\enquote{the symbols used in Section 2 where not previously explained}}

All symbols in Section 2 are explained in that section.
As suggested by Reviewer 2, we have improved some of the notations.

\textbf{\enquote{How was the threshold of 200 chosen?}}

This threshold was defined empirically.
This has been clarified in Section 8.3.1.

\textbf{\enquote{The proof of correctness in section 7.1.4 is strange and it is not really a formal proof}}

We have rewritten the proof to use induction, which makes it more formal.

\end{letter}
\end{document}
