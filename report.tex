\let\oldvec\vec % prevent redefinition of vec for amssymb
\documentclass[sw]{iosart2c}
\let\vec\oldvec
\usepackage[utf8]{inputenc}

\usepackage[numbers]{natbib}% for bibliography sorting/compressing
%\usepackage{amsmath}
%\usepackage{endnotes}
%\usepackage{graphics}

%%%%%%%%%%% Put your definitions here

% Language
\usepackage[english]{babel}

% Typography
\usepackage[T1]{fontenc}
\usepackage{newtxtext,newtxmath}
\usepackage{microtype}
\usepackage{slantsc}
\usepackage[scaled=.8]{beramono} % Bera for fixed-width text
\usepackage{csquotes}

\makeatletter
% Make paragraph stand out better with a bold label
\renewcommand\paragraph{\@startsection{paragraph}{4}{\z@}%
                       {-12\p@ \@plus -4\p@ \@minus -4\p@}%
                       {-0.5em \@plus -0.22em \@minus -0.1em}%
                       {\normalfont\normalsize\bfseries}}
\makeatother

% OOOOO Circles OOOO
\usepackage{wasysym}

% Math
\usepackage{amsmath}
\usepackage{amssymb}
\usepackage{mathtools}
\newcommand*{\defeq}{\mathrel{\vcenter{\baselineskip0.45ex \lineskiplimit0pt
                     \hbox{\scriptsize.}\hbox{\scriptsize.}}}=}
\newenvironment{Definition}{%
  \vspace{-.25\baselineskip}%
  \begin{definition}%
}
{%
  \end{definition}%
  \vspace{-.25\baselineskip}%
}
\makeatletter
\newcommand{\raisemath}[1]{\mathpalette{\raisem@th{#1}}}
\newcommand{\raisem@th}[3]{\raisebox{#1}{$#2#3$}}
\makeatother

% Lists
\usepackage{enumitem}
\setitemize    {parsep=0pt,itemsep=1pt,topsep=1pt}
\setenumerate  {parsep=0pt,itemsep=1pt,topsep=1pt}
\setdescription{parsep=0pt,itemsep=2pt,topsep=1pt}
\newlist{inlinelist}{enumerate*}{1}
\setlist[inlinelist,1]{label=\roman*)}

% Tables
\usepackage{multirow,booktabs,tabularx}

% Code
\usepackage{fancyvrb}
\let\verb\Verb
\usepackage{listings}
\lstset{
    captionpos=b,
    columns=fullflexible,
    breaklines=true,
    basicstyle=\ttfamily\fontsize{9}{10}\selectfont,
    escapechar=§,
    abovecaptionskip=0em,
    belowcaptionskip=0em,
    aboveskip=0em,
    belowskip=0em,
    keepspaces = true,
}
\lstdefinestyle{inline}{
  aboveskip=.3em,
  belowskip=.3em,
  xleftmargin=\parindent,
}

\usepackage[algo2e,linesnumbered,noend]{algorithm2e}
\usepackage{verbatim}
\usepackage{framed}
\newcommand{\code}[1]{\lstinline[basicstyle=\ttfamily]{#1}}
\SetKw{InlineIf}{if}
\SetKwProg{Function}{Function}{}{end}
\renewcommand\AlCapSty{\footnotesize}
\renewcommand\AlCapFnt{\normalfont}
\makeatletter
% Call algorithms "Listing"
\newcounter{megaalgorithm}
\newenvironment{listingalgorithm}[1][htb]
  {\renewcommand{\algorithmname}{Listing}
   \begin{algorithm2e*}[#1]%
  }{\end{algorithm2e*}}
\makeatother

% Fix large spacing above each footnote
\usepackage[hang]{footmisc}
\setlength{\footnotemargin}{4mm}

% References
\usepackage{natbib}
\usepackage[hyphens]{url}
\usepackage[colorlinks=false,linkcolor=black,citecolor=black,pdfpagelabels=false]{hyperref}
\usepackage[capitalize]{cleveref}
\crefname{line}{line}{lines}

% Plots
\usepackage[dvipsnames,svgnames]{xcolor}
\usepackage{tikz}
\usetikzlibrary{arrows,positioning,shapes,calc}
\usepackage{pgfplots}
\definecolor{one}  {RGB}{142, 23,  4}
\definecolor{two}  {RGB}{ 62,111,186}
\definecolor{three}{RGB}{172,196, 75}
\newcommand\plotfontsize{\fontsize{6}{6}\selectfont}
\pgfplotsset{%
  compat=1.6,
  xmin=0.9, xmax=270,
  axis lines=left,
  every axis/.append style={
    font=\plotfontsize,
  },
  label style={
    font=\plotfontsize\bfseries,
  },
  tick label style={
    font=\plotfontsize,
  },
  legend cell align=left,
  legend style={
    /tikz/every even column/.append style={column sep=.3em},
    draw=none, fill=none,
    inner sep=0pt, outer sep=0pt,
    anchor=north east,
    text height=3pt,
  },
  legend image post style={only marks},
  log base 10 number format code/.code={%
    $\pgfmathparse{10^(#1)}\pgfmathprintnumber{\pgfmathresult}$%
  },
  % Don't show axis exponent
  ytick scale label code/.code={},
}

% Graphics
\usepackage{subcaption}
\captionsetup[figure]{font=small,labelfont=bf}
\captionsetup[lstlisting]{font=small,labelfont=bf}
\DeclareCaptionLabelFormat{bf-parens}{\textbf{Fig.\thinspace#2}}
\captionsetup[subfigure]{labelformat=simple,labelsep=colon,font=scriptsize,labelformat=bf-parens}
\usetikzlibrary{calc,positioning,arrows,matrix,fit,backgrounds}
\setlength\abovecaptionskip{.5em plus .5em}
\setlength\textfloatsep{1.5em plus .5em}
\renewcommand\bottomfraction{0.5}
\renewcommand\textfraction{0.05}
\renewcommand\topfraction{0.9}
\graphicspath{{./img/}}

% Research questions
\newcounter{researchquestion}
\newenvironment{researchquestion}
  {\par\addvspace{.5\baselineskip minus .3\baselineskip}%
   \refstepcounter{researchquestion}%
   \noindent{\bfseries Question \theresearchquestion:} }
  {\par\addvspace{.5\baselineskip minus .3\baselineskip}}
\crefname{researchquestion}{question}{questions}

\newcounter{hypothesis}
\newenvironment{hypothesis}
  {\par\addvspace{.5\baselineskip minus .3\baselineskip}%
   \refstepcounter{hypothesis}%
   \noindent{\bfseries Hypothesis \thehypothesis:} }
  {\par\addvspace{.5\baselineskip minus .3\baselineskip}}
\crefname{hypothesis}{Hypothesis}{Hypotheses}

\newcounter{task}
\newenvironment{task}
  {\par\addvspace{.5\baselineskip minus .3\baselineskip}%
   \refstepcounter{task}%
   \noindent{\bfseries Task \thetask:} }
  {\par\addvspace{.5\baselineskip minus .3\baselineskip}}
\crefname{task}{Task}{Tasks}

\newcounter{frequirement}
\newenvironment{frequirement}
  {\par\addvspace{.5\baselineskip minus .3\baselineskip}%
   \refstepcounter{frequirement}%
   \noindent{\bfseries Functional Requirement \thefrequirement:} }
  {\par\addvspace{.5\baselineskip minus .3\baselineskip}}
\crefname{frequirement}{Functional Requirement}{Functional Requirements}

\newcounter{nfrequirement}
\newenvironment{nfrequirement}
  {\par\addvspace{.5\baselineskip minus .3\baselineskip}%
   \refstepcounter{nfrequirement}%
   \noindent{\bfseries Non-Functional Requirement \thenfrequirement:} }
  {\par\addvspace{.5\baselineskip minus .3\baselineskip}}
\crefname{nfrequirement}{Non-Functional Requirement}{Non-Functional Requirements}

\Crefname{pluralequation}{Equations}{Equations}
\Crefformat{equation}{#2Equation~#1#3}
\Crefname{figure}{Figure}{Figures}

% Indication of to-dos
% Comments
\usepackage[normalem]{ulem}
\makeatletter
\font\uwavefont=lasyb10 scaled 700
\def\spelling{\bgroup\markoverwith{\lower3.5\p@\hbox{\uwavefont\textcolor{Red}{\char58}}}\ULon}
\def\grammar{\bgroup\markoverwith{\lower3.5\p@\hbox{\uwavefont\textcolor{LimeGreen}{\char58}}}\ULon}
\def\phrasing{\bgroup\markoverwith{\lower3.5\p@\hbox{\uwavefont\textcolor{RoyalBlue}{\char58}}}\ULon}
\let\rephrase\phrasing
\newcommand\remove{\bgroup\markoverwith{\textcolor{red}{\rule[0.5ex]{2pt}{0.4pt}}}\ULon}
\makeatother
\newcommand{\todo}[1]{\noindent\textcolor{red}{{\bf \{TODO}: #1{\bf \}}}}
\newcommand{\TODO}[1]{\todo{#1}}
\newcommand{\citeneeded}{\textcolor{red}{{\bf [?!]}}}
\newenvironment{draft}{\color{gray}}{\color{black}}
\newcommand{\rv}[1]{{\color{RubineRed}\textbf{RV}: #1}}
\newcommand{\rt}[1]{\noindent\textcolor{red}{{\bf \{RT}: #1{\bf \}}}}
\newcommand{\pc}[1]{\noindent\textcolor{red}{{\bf \{PC}: #1{\bf \}}}}

% Acronyms
\usepackage{xspace}
\newcommand\Acronym[1]{%
  \expandafter\def\csname#1\endcsname{{\scshape #1}\xspace}%
  \expandafter\def\csname#1s\endcsname{{\scshape #1}s\xspace}%
}
\Acronym{rdf}
\Acronym{sparql}
\Acronym{gtfs}
\Acronym{csv}
\Acronym{zip}
\Acronym{api}
\Acronym{cpu}
\Acronym{json}
\Acronym{npm}
\Acronym{rsp}
\Acronym{hobbit}
\Acronym{eswc}
\Acronym{uri}
\newcommand\gtfslc{{\scshape gtfs}2{\scshape lc}\xspace}
\newcommand\podigg{{\scshape p}o{\scshape d}i{\scshape gg}\xspace}
\newcommand\podigglc{\podigg-{\scshape lc}\xspace}

%%%%%%%%%%%%%%%%%%%%%%%%%%%%%%%%%%%
%% Symbols for math formulas
%%%%%%%%%%%%%%%%%%%%%%%%%%%%%%%%%%%

%%%%%%%%%%%%%%%%%%%%%%%%%%%%%%%%%%%
%%% Basics
\newcommand{\definedAs}    % the symbol for defining the thing on the LHS
	%{\mathrel{\mathop:}=}
	{=}

%%%%%%%%%%%%%%%%%%%%%%%%%%%%%%%%%%%


\crefname{researchquestion}{Research Question}{Research Questions}

% Fix and style captions
\usepackage{microtype}
\usepackage{caption}
\DeclareCaptionFont{ls}{\lsstyle}
\captionsetup{
  format = plain,
  font = footnotesize,
  labelfont = {bf},
}

%%%%%%%%%%% End of definitions

\pubyear{0000}
\volume{0}
\firstpage{1}
\lastpage{1}

\begin{document}

\begin{frontmatter}

%\pretitle{}
\title{Triple Store for Changesets}
%\subtitle{}

%\review{}{}{}


\author{\fnms{Ruben} \snm{Taelman}}\sep
\author{\fnms{Miel} \snm{Vander Sande}}\sep
\author{\fnms{Ruben} \snm{Verborgh}}\and
\author{\fnms{Erik} \snm{Mannens}}
\address{imec - Ghent University - IDLab, Sint-Pietersnieuwstraat 41, 9000 Ghent, Belgium\\
E-mail: \url{ruben.taelman@ugent.be}}
\runningauthor{Ruben Taelman}

\renewcommand\abstractname{\textbf{Abstract}.\\}
\begin{abstract}
    % Context
    % Need
    % Task
    % Object
    % Findings
    % Conclusion
\end{abstract}

\begin{keyword}
    
\end{keyword}

\end{frontmatter}

\begin{section}{Introduction}
    \label{sec:Introduction}
    %\input{section-introduction.tex}
\end{section}

\begin{section}{Related Work}
    \label{sec:RelatedWork}
    %\input{section-related-work.tex}
\end{section}

\begin{section}{Research Question}
    \label{sec:ResearchQuestion}
    %\input{section-research-question.tex}
\end{section}

\begin{section}{Method}
    \label{sec:Methodology}
    %\input{section-methodology.tex}
\end{section}

\begin{section}{Implementation}
    \label{sec:Implementation}
    %\input{section-implementation.tex}
\end{section}

\begin{section}{Evaluation}
    \label{sec:Evaluation}
    \begin{subsection}{Result}
    \label{subsec:evaluation:results}
    
    \definecolor{colorvm}{RGB}{142, 23,  4}
    \definecolor{colordm}{RGB}{ 62,111,186}
    \definecolor{colorvq}{RGB}{172,196, 75}

    \pgfplotsset{
        main-raw/.style={every mark/.append style={solid, fill=gray},thick},
        main/.style={smooth,main-raw},
        query1/.style={color=colorvm,main,mark=*,mark options={solid,fill=colorvm},mark size=0.3},
        query2/.style={color=colordm,main,mark=square*,mark options={solid,fill=colordm},mark size=0.3},
        query3/.style={color=colorvq,main,mark=triangle*,mark options={solid,fill=colorvq},mark size=0.3},
        main-bar/.style={},
        query1-bar/.style={black,fill=colorvm,main-bar},
        query2-bar/.style={black,fill=colordm,main-bar},
        query3-bar/.style={black,fill=colorvq,main-bar},
        avg/.style={},
        max/.style={dashed},
        min/.style={dashed},
        query-dur-base/.style={%
            xmin=0,
            xmax=9,
            ylabel={Lookup (ms)},
            width=6cm,
            height=4cm,
            every axis legend/.append style={at={(0.6,1)},anchor=north},
            max space between ticks=30pt,try min ticks=2,
            %y label style={at={(-55pt,50pt)}, rotate=-90, anchor=west},
            legend columns=3,
        },
        query-dur/.style={%
            query-dur-base,
            xlabel={Version},
            legend image post style={sharp plot, line width=0.5pt},
            x label style={at={(0pt,-30pt)}, anchor=north west},
        },
        query-dur-bar/.style={%
            query-dur-base,
            ybar,
            legend image code/.code={%
\draw[##1,/tikz/.cd,bar width=3pt,yshift=-0.2em,bar shift=0pt]
plot coordinates {(0cm,0.8em) (2*\pgfplotbarwidth,0.6em)};},
            xmajorticks=false,
            xlabel={},
        },
        discard if below/.style 2 args={
            x filter/.append code={
                \edef\tempa{\thisrow{#1}}
                \edef\tempb{#2}
                \ifdim\tempa pt<\tempb pt
                \else
                    \def\pgfmathresult{inf}
                \fi
            }
        },
        discard if not/.style 2 args={
            x filter/.append code={
                \edef\tempa{\thisrow{#1}}
                \edef\tempb{#2}
                \ifx\tempa\tempb
                \else
                    \def\pgfmathresult{inf}
                \fi
            }
        }
    }
    
    \begin{figure*}[p]
      \begin{subfigure}{0.3\textwidth}
          \centering
          \input{img/query-dur-vm-low.tex}
          \caption{Low result cardinality}
          \label{fig:query-dur-vm-low}
      \end{subfigure}
      \hspace{.022\textwidth}
      \begin{subfigure}{0.3\textwidth}
          \centering
          \begin{tikzpicture}
    \begin{axis}[ymode=log,query-dur,ymin=0,ymax=100,legend entries = {
            Subject 1,Predicate 1,Object 1,
            Subject all,Predicate all,Object all
        }]        
        \addplot[query1,max,discard if not={offset}{0}] table [x=patch, y=lookup-mus-1, col sep=comma] {data/s-high/_average_vm.csv};
        \addplot[query2,max,discard if not={offset}{0}] table [x=patch, y=lookup-mus-1, col sep=comma] {data/p-high/_average_vm.csv};
        \addplot[query3,max,discard if not={offset}{0}] table [x=patch, y=lookup-mus-1, col sep=comma] {data/o-high/_average_vm.csv};
        
        \addplot[query1,avg,discard if not={offset}{0}] table [x=patch, y=lookup-mus-inf, col sep=comma] {data/s-high/_average_vm.csv};
        \addplot[query2,avg,discard if not={offset}{0}] table [x=patch, y=lookup-mus-inf, col sep=comma] {data/p-high/_average_vm.csv};
        \addplot[query3,avg,discard if not={offset}{0}] table [x=patch, y=lookup-mus-inf, col sep=comma] {data/o-high/_average_vm.csv};
    \end{axis}
\end{tikzpicture}
          \caption{High result cardinality}
          \label{fig:query-dur-vm-high}
      \end{subfigure}
      \caption{Average lookup times for VM queries}
      \label{fig:query-dur-vm}
    \end{figure*}
    
    \begin{figure*}[p]
      \begin{subfigure}{0.3\textwidth}
          \centering
          \begin{tikzpicture}
    \begin{axis}[ymode=log,query-dur,xmin=1,ymin=0,ymax=100,legend entries = {
            Subject 1,Predicate 1,Object 1,
            Subject all,Predicate all,Object all
        }]        
        \addplot[query1,max,discard if not={offset}{0},discard if not={patch_start}{0}] table [x=patch_end, y=lookup-mus-1, col sep=comma] {data/s-low/_average_dm.csv};
        \addplot[query2,max,discard if not={offset}{0},discard if not={patch_start}{0}] table [x=patch_end, y=lookup-mus-1, col sep=comma] {data/p-low/_average_dm.csv};
        \addplot[query3,max,discard if not={offset}{0},discard if not={patch_start}{0}] table [x=patch_end, y=lookup-mus-1, col sep=comma] {data/o-low/_average_dm.csv};
        
        \addplot[query1,avg,discard if not={offset}{0},discard if not={patch_start}{0}] table [x=patch_end, y=lookup-mus-inf, col sep=comma] {data/s-low/_average_dm.csv};
        \addplot[query2,avg,discard if not={offset}{0},discard if not={patch_start}{0}] table [x=patch_end, y=lookup-mus-inf, col sep=comma] {data/p-low/_average_dm.csv};
        \addplot[query3,avg,discard if not={offset}{0},discard if not={patch_start}{0}] table [x=patch_end, y=lookup-mus-inf, col sep=comma] {data/o-low/_average_dm.csv};
    \end{axis}
\end{tikzpicture}
          \caption{Low result cardinality}
          \label{fig:query-dur-dm-low}
      \end{subfigure}
      \hspace{.022\textwidth}
      \begin{subfigure}{0.3\textwidth}
          \centering
          \input{img/query-dur-dm-high.tex}
          \caption{High result cardinality}
          \label{fig:query-dur-dm-high}
      \end{subfigure}
      \caption{Average lookup times for DM queries from version 0 to all other versions}
      \label{fig:query-dur-dm}
    \end{figure*}
    
    \begin{table}[b]
        \centering
        \DTLloaddb{slow}{data/s-low/_average_vq.csv}
        \DTLloaddb{shigh}{data/s-high/_average_vq.csv}
        \DTLloaddb{plow}{data/p-low/_average_vq.csv}
        \DTLloaddb{phigh}{data/p-high/_average_vq.csv}
        \DTLloaddb{olow}{data/o-low/_average_vq.csv}
        \DTLloaddb{ohigh}{data/o-high/_average_vq.csv}
        \DTLloaddb{polow}{data/po-low/_average_vq.csv}
        \DTLloaddb{pohigh}{data/po-high/_average_vq.csv}
        \DTLloaddb{solow}{data/so-low/_average_vq.csv}
        \DTLloaddb{splow}{data/sp-low/_average_vq.csv}
        \DTLloaddb{sphigh}{data/sp-high/_average_vq.csv}
        \DTLloaddb{spo}{data/spo/_average_vq.csv}
        \begin{tabularx}{\linewidth}{l|rr}
            & \textbf{Low cardinality} & \textbf{High cardinality} \\
            \hline
            \textbf{S 1} & \DTLforeach*{slow}{\v=lookup-mus-1}{\v} & \DTLforeach*{shigh}{\v=lookup-mus-1}{\v} \\
            \textbf{S all} & \DTLforeach*{slow}{\v=lookup-mus-inf}{\v} & \DTLforeach*{shigh}{\v=lookup-mus-inf}{\v} \\
            \textbf{P 1} & \DTLforeach*{plow}{\v=lookup-mus-1}{\v} & \DTLforeach*{phigh}{\v=lookup-mus-1}{\v} \\
            \textbf{P all} & \DTLforeach*{plow}{\v=lookup-mus-inf}{\v} & \DTLforeach*{phigh}{\v=lookup-mus-inf}{\v} \\
            \textbf{O 1} & \DTLforeach*{olow}{\v=lookup-mus-1}{\v} & \DTLforeach*{ohigh}{\v=lookup-mus-1}{\v} \\
            \textbf{O all} & \DTLforeach*{olow}{\v=lookup-mus-inf}{\v} & \DTLforeach*{ohigh}{\v=lookup-mus-inf}{\v} \\
            \textbf{PO 1} & \DTLforeach*{polow}{\v=lookup-mus-1}{\v} & \DTLforeach*{pohigh}{\v=lookup-mus-1}{\v} \\
            \textbf{PO all} & \DTLforeach*{polow}{\v=lookup-mus-inf}{\v} & \DTLforeach*{pohigh}{\v=lookup-mus-inf}{\v} \\
            \textbf{SO 1} & \DTLforeach*{solow}{\v=lookup-mus-1}{\v} &  \\
            \textbf{SO all} & \DTLforeach*{solow}{\v=lookup-mus-inf}{\v} &  \\
            \textbf{SP 1} & \DTLforeach*{splow}{\v=lookup-mus-1}{\v} & \DTLforeach*{sphigh}{\v=lookup-mus-1}{\v} \\
            \textbf{SP all} & \DTLforeach*{splow}{\v=lookup-mus-inf}{\v} & \DTLforeach*{sphigh}{\v=lookup-mus-inf}{\v} \\
            \textbf{SPO} & \DTLforeach*{spo}{\v=lookup-mus-1}{\v} &  \\
        \end{tabularx}
        \caption{Average lookup times for VQ queries in milliseconds}
        \label{fig:query-dur-dm}
    \end{table}

\end{subsection}

\end{section}

\begin{section}{Discussion}
    \label{sec:Discussion}
    %\input{section-discussion.tex}
\end{section}

\begin{section}{Conclusions}
    \label{sec:Conclusions}
    %\input{section-conclusions.tex}
\end{section}

\bibliographystyle{abbrv}
\bibliography{references}

\end{document}
